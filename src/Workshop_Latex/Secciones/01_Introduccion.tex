\section{Introduction}

The project consists of a web-based platform that collects, processes, and presents air-quality information using relational and scalable data-management techniques. Its primary motivation is to provide reliable, uniform, and accessible environmental data that can support public decision-making, research activities, and citizen awareness. Air pollution directly affects health, mobility, and long-term urban planning, making timely and well-structured information essential for institutions and the general population.

The goal of the platform is to integrate heterogeneous external data sources, transform them into a consistent operational dataset, and expose processed information through dashboards, historical queries, reports, and alerts. The scope of this project is strictly limited to the data-management layer: data ingestion, storage, normalization, querying, and visualization of core indicators. Advanced features such as predictive analytics, recommendation engines, mobile applications, and geographic exploration are considered out of scope for the current deliverable and are not included unless required for demonstrating database concepts.

The system focuses on demonstrating ingestion pipelines, a hybrid storage model (SQL + structured raw data), performance-oriented query design, and a basic user interaction layer for consuming processed information. This allows the project to remain aligned with the learning objectives of the course while providing a realistic foundation for future extensions.
