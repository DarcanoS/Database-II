% User stories updated according to instructor feedback.
% Source: _temp_corrections/User_Stories.csv and User_Stories_changes_spec.md

\section{User Stories}

User stories describe the functional needs of the platform from the perspective of different stakeholder roles. Each story follows the standard \emph{As a / I want to / So that} structure, and includes priority, effort estimates, and explicit acceptance criteria. The stories are aligned with the refined functional and non-functional requirements presented in the previous section, focusing on the realistic baseline scope of the project.

% ========================================
% US1 - Automated Data Ingestion
% ========================================
\subsubsection*{US1 -- Automated Data Ingestion}

\begin{table}[H]
  \centering
  \begin{tabularx}{\textwidth}{|p{0.20\textwidth}|X|}
    \hline
    \textbf{As a} & Technical Administrator \\
    \hline
    \textbf{I want to} & Collect up-to-date air quality data from external providers in an automated way \\
    \hline
    \textbf{So that} & The platform can provide accurate and current information on air quality without manual imports \\
    \hline
    \textbf{Priority} & Must \\
    \hline
    \textbf{Effort} & 8 \\
    \hline
    \textbf{Acceptance criteria} & Ingestion job runs on a configurable schedule (e.g., every 10-60 minutes); At least one external provider is ingested without manual intervention; Failed ingestions are logged with error details; Successful runs are logged with timestamps; Ingested readings appear in the database within the expected delay after the external API update \\
    \hline
  \end{tabularx}
  \caption{User story US1}
  \label{tab:user_story_us1}
\end{table}

% ========================================
% US2 - Filterable Historical Data Access
% ========================================
\subsubsection*{US2 -- Filterable Historical Data Access}

\begin{table}[H]
  \centering
  \begin{tabularx}{\textwidth}{|p{0.20\textwidth}|X|}
    \hline
    \textbf{As a} & Researcher/Analyst \\
    \hline
    \textbf{I want to} & Access historical air quality data filtered by city, date range, and pollutant \\
    \hline
    \textbf{So that} & I can perform longitudinal analysis and scientific research \\
    \hline
    \textbf{Priority} & Must \\
    \hline
    \textbf{Effort} & 5 \\
    \hline
    \textbf{Acceptance criteria} & User can select city, date range, and pollutant from the interface; System returns matching records from historical data; Results can be exported to CSV and JSON; At least 3 years of historical data are available for test cities \\
    \hline
  \end{tabularx}
  \caption{User story US2}
  \label{tab:user_story_us2}
\end{table}

% ========================================
% US3 - Efficient Analytical Queries
% ========================================
\subsubsection*{US3 -- Efficient Analytical Queries}

\begin{table}[H]
  \centering
  \begin{tabularx}{\textwidth}{|p{0.20\textwidth}|X|}
    \hline
    \textbf{As a} & Technical Administrator \\
    \hline
    \textbf{I want to} & Run queries over large volumes of air quality data without significant delays \\
    \hline
    \textbf{So that} & I can support analysts and dashboards without performance bottlenecks \\
    \hline
    \textbf{Priority} & Should \\
    \hline
    \textbf{Effort} & 5 \\
    \hline
    \textbf{Acceptance criteria} & Typical dashboard queries over AirQualityDailyStats complete in under 2 seconds for test datasets; Historical queries over several months complete in under 5 seconds; Database indexes for common filters (station, date, pollutant) are documented and enabled \\
    \hline
  \end{tabularx}
  \caption{User story US3}
  \label{tab:user_story_us3}
\end{table}

% ========================================
% US4 - Dashboards with Key Indicators
% ========================================
\subsubsection*{US4 -- Dashboards with Key Indicators}

\begin{table}[H]
  \centering
  \begin{tabularx}{\textwidth}{|p{0.20\textwidth}|X|}
    \hline
    \textbf{As a} & Public Policy Manager \\
    \hline
    \textbf{I want to} & View dashboards with key air quality indicators for selected regions \\
    \hline
    \textbf{So that} & I can quickly understand current conditions and recent trends to inform decisions \\
    \hline
    \textbf{Priority} & Must \\
    \hline
    \textbf{Effort} & 8 \\
    \hline
    \textbf{Acceptance criteria} & Dashboard shows at least: current AQI, main pollutant, and daily trend for the selected city or station; Dashboard updates when the user changes city or station; Dashboard uses AirQualityDailyStats for historical trends and raw readings for current values \\
    \hline
  \end{tabularx}
  \caption{User story US4}
  \label{tab:user_story_us4}
\end{table}

% ========================================
% US5 - Custom Report Generation
% ========================================
\subsubsection*{US5 -- Custom Report Generation}

\begin{table}[H]
  \centering
  \begin{tabularx}{\textwidth}{|p{0.20\textwidth}|X|}
    \hline
    \textbf{As a} & Researcher/Analyst \\
    \hline
    \textbf{I want to} & Generate custom reports with filters and download them \\
    \hline
    \textbf{So that} & I can use the data in external tools or include it in my own analyses \\
    \hline
    \textbf{Priority} & Must \\
    \hline
    \textbf{Effort} & 5 \\
    \hline
    \textbf{Acceptance criteria} & User can configure a report by choosing city/region, date range, and pollutants; System generates the report and confirms when it is ready; User can download the report in at least CSV format \\
    \hline
  \end{tabularx}
  \caption{User story US5}
  \label{tab:user_story_us5}
\end{table}

% ========================================
% US6 - Time-Series Graphs
% ========================================
\subsubsection*{US6 -- Time-Series Graphs}

\begin{table}[H]
  \centering
  \begin{tabularx}{\textwidth}{|p{0.20\textwidth}|X|}
    \hline
    \textbf{As a} & Citizen \\
    \hline
    \textbf{I want to} & See simple graphs of how air quality changes over time in my city \\
    \hline
    \textbf{So that} & I can understand whether conditions are improving or getting worse \\
    \hline
    \textbf{Priority} & Must \\
    \hline
    \textbf{Effort} & 3 \\
    \hline
    \textbf{Acceptance criteria} & User can select a city and a pollutant; System displays a time-series chart for the selected period; User can change the date range (e.g., last 7 days vs last 30 days) and the chart updates accordingly \\
    \hline
  \end{tabularx}
  \caption{User story US6}
  \label{tab:user_story_us6}
\end{table}

% ========================================
% US7 - Simple Rule-Based Recommendations
% ========================================
\subsubsection*{US7 -- Simple Rule-Based Recommendations}

\begin{table}[H]
  \centering
  \begin{tabularx}{\textwidth}{|p{0.20\textwidth}|X|}
    \hline
    \textbf{As a} & Citizen \\
    \hline
    \textbf{I want to} & Receive simple recommendations based on current air quality at my location \\
    \hline
    \textbf{So that} & I can protect my health when air quality is poor \\
    \hline
    \textbf{Priority} & Should \\
    \hline
    \textbf{Effort} & 5 \\
    \hline
    \textbf{Acceptance criteria} & User can set a default city or location; When AQI exceeds defined thresholds, the system shows recommendations such as avoiding outdoor exercise or using a mask; Recommendations are based on simple, documented rules (no complex machine learning required) \\
    \hline
  \end{tabularx}
  \caption{User story US7}
  \label{tab:user_story_us7}
\end{table}

% ========================================
% US8 - Configurable Alerts
% ========================================
\subsubsection*{US8 -- Configurable Alerts}

\begin{table}[H]
  \centering
  \begin{tabularx}{\textwidth}{|p{0.20\textwidth}|X|}
    \hline
    \textbf{As a} & Citizen \\
    \hline
    \textbf{I want to} & Configure alerts when air quality exceeds a certain threshold \\
    \hline
    \textbf{So that} & I can be notified when conditions become unhealthy \\
    \hline
    \textbf{Priority} & Must \\
    \hline
    \textbf{Effort} & 5 \\
    \hline
    \textbf{Acceptance criteria} & User can create an alert selecting city or station, pollutant, and AQI threshold; User can choose at least one notification channel (e.g., email or in-app notification); When AQI exceeds the configured threshold, an alert is recorded and shown; User can deactivate or delete existing alerts \\
    \hline
  \end{tabularx}
  \caption{User story US8}
  \label{tab:user_story_us8}
\end{table}

% ========================================
% US9 - Informational Protective Measures (Optional)
% ========================================
\subsubsection*{US9 -- Informational Protective Measures (Optional)}

\begin{table}[H]
  \centering
  \begin{tabularx}{\textwidth}{|p{0.20\textwidth}|X|}
    \hline
    \textbf{As a} & Citizen \\
    \hline
    \textbf{I want to} & See informational suggestions about protective measures during high pollution episodes \\
    \hline
    \textbf{So that} & I can decide whether to use masks, air purifiers, or other measures \\
    \hline
    \textbf{Priority} & Could \\
    \hline
    \textbf{Effort} & 3 \\
    \hline
    \textbf{Acceptance criteria} & When AQI is above a defined level, the interface shows text with recommended protective measures; Suggestions are informational only and do not require e-commerce links \\
    \hline
  \end{tabularx}
  \caption{User story US9}
  \label{tab:user_story_us9}
\end{table}

% ========================================
% US10 - Geographic Search
% ========================================
\subsubsection*{US10 -- Geographic Search}

\begin{table}[H]
  \centering
  \begin{tabularx}{\textwidth}{|p{0.20\textwidth}|X|}
    \hline
    \textbf{As a} & Citizen \\
    \hline
    \textbf{I want to} & Search air quality by country, city, or region \\
    \hline
    \textbf{So that} & I can compare air quality in different places \\
    \hline
    \textbf{Priority} & Must \\
    \hline
    \textbf{Effort} & 3 \\
    \hline
    \textbf{Acceptance criteria} & User can search by country and see available cities or stations; User can search directly by city name and open its dashboard; If regions are defined, the user can filter stations by region \\
    \hline
  \end{tabularx}
  \caption{User story US10}
  \label{tab:user_story_us10}
\end{table}

% ========================================
% US11 - Responsive Web Interface
% ========================================
\subsubsection*{US11 -- Responsive Web Interface}

\begin{table}[H]
  \centering
  \begin{tabularx}{\textwidth}{|p{0.20\textwidth}|X|}
    \hline
    \textbf{As a} & Citizen \\
    \hline
    \textbf{I want to} & Access the platform from different devices (mobile, tablet, desktop) \\
    \hline
    \textbf{So that} & I can check air quality whenever I need it, from any device \\
    \hline
    \textbf{Priority} & Must \\
    \hline
    \textbf{Effort} & 5 \\
    \hline
    \textbf{Acceptance criteria} & Core views (home, dashboard, alerts) are usable on mobile, tablet, and desktop browsers; Layout adapts without breaking text or hiding essential information; No native mobile app is required, the responsive web app is sufficient \\
    \hline
  \end{tabularx}
  \caption{User story US11}
  \label{tab:user_story_us11}
\end{table}

% ========================================
% US12 - Sharing Views (Optional)
% ========================================
\subsubsection*{US12 -- Sharing Views (Optional)}

\begin{table}[H]
  \centering
  \begin{tabularx}{\textwidth}{|p{0.20\textwidth}|X|}
    \hline
    \textbf{As a} & Citizen \\
    \hline
    \textbf{I want to} & Share air quality views with other people through social media or messaging \\
    \hline
    \textbf{So that} & I can raise awareness about air quality conditions \\
    \hline
    \textbf{Priority} & Could \\
    \hline
    \textbf{Effort} & 3 \\
    \hline
    \textbf{Acceptance criteria} & User can obtain a shareable link to the current view or report; Link opens the same view for other users without requiring them to reapply filters \\
    \hline
  \end{tabularx}
  \caption{User story US12}
  \label{tab:user_story_us12}
\end{table}

% ========================================
% US13 - Fast User Experience
% ========================================
\subsubsection*{US13 -- Fast User Experience}

\begin{table}[H]
  \centering
  \begin{tabularx}{\textwidth}{|p{0.20\textwidth}|X|}
    \hline
    \textbf{As a} & Citizen \\
    \hline
    \textbf{I want to} & Experience fast loading times when using the platform \\
    \hline
    \textbf{So that} & I do not abandon the platform due to slow responses \\
    \hline
    \textbf{Priority} & Must \\
    \hline
    \textbf{Effort} & 5 \\
    \hline
    \textbf{Acceptance criteria} & Under normal load, main dashboards load in under 2 seconds for test users; Pagination or lazy loading is used where necessary to avoid rendering very large lists at once \\
    \hline
  \end{tabularx}
  \caption{User story US13}
  \label{tab:user_story_us13}
\end{table}

% ========================================
% US14 - Monitoring Ingestion Jobs
% ========================================
\subsubsection*{US14 -- Monitoring Ingestion Jobs}

\begin{table}[H]
  \centering
  \begin{tabularx}{\textwidth}{|p{0.20\textwidth}|X|}
    \hline
    \textbf{As a} & Technical Administrator \\
    \hline
    \textbf{I want to} & Monitor ingestion jobs and detect failures \\
    \hline
    \textbf{So that} & I can quickly react if external providers change or if ingestion stops \\
    \hline
    \textbf{Priority} & Should \\
    \hline
    \textbf{Effort} & 5 \\
    \hline
    \textbf{Acceptance criteria} & There is a view or log where the status of recent ingestion jobs is visible; For each job, the system stores timestamp, data source, and result (success or failure); Failure entries include a brief error message to guide debugging \\
    \hline
  \end{tabularx}
  \caption{User story US14}
  \label{tab:user_story_us14}
\end{table}

\newpage