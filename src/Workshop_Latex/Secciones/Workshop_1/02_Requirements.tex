\section{Requirements}

% Functional requirements (FR1–FR14) updated according to instructor feedback – see Requirements_FR_corrections.csv

\begin{longtable}{|p{2cm}|p{2cm}|p{8cm}|p{3cm}|}
\hline
\textbf{ID} & \textbf{Type} & \textbf{Requirement} & \textbf{Associated User Stories} \\
\hline
\endfirsthead

\hline
\textbf{ID} & \textbf{Type} & \textbf{Requirement} & \textbf{Associated User Stories} \\
\hline
\endhead

FR1 & Functional & The system must periodically collect up-to-date air quality data from multiple external sources (e.g., APIs and monitoring stations) and store it for further processing. & US1, US13, US14 \\ 
\hline

FR2 & Functional & The system must allow users to query and visualize historical air quality data filtered by date range, location, and pollutant type. & US2, US7 \\ 
\hline

FR3 & Functional & The system must display air quality information in a consistent and clear format, independently of the original data source. & US1, US3 \\ 
\hline

FR4 & Functional & The system must display dashboards with key air quality indicators (e.g., AQI, main pollutant, daily trends) based on the latest available data. & US4 \\ 
\hline

FR5 & Functional & The system must generate custom reports with filters for date range, location, and selected indicators, and allow users to download these reports in standard formats. & US5 \\ 
\hline

FR6 & Functional & The system must present graphs showing the evolution of air quality over time, allowing users to interact with basic controls such as selecting date ranges or toggling pollutants. & US6 \\ 
\hline

FR7 & Functional & The system must allow users to export historical air quality data in standard formats such as CSV and JSON, based on the filters applied in the user interface. & US7 \\ 
\hline

FR8 & Functional & The system must provide simple, rule-based recommendations based on the user's location and current air quality conditions (e.g., avoiding outdoor exercise, using a mask). & US8 \\ 
\hline

FR9 & Functional & The system must send air quality alerts when configurable thresholds are exceeded, according to each user's notification preferences. & US9 \\ 
\hline

FR10 & Functional & \textit{(Optional)} The system may suggest appropriate protective measures and product categories (e.g., certified masks, air purifiers) during high pollution periods, as informational guidance only. & US10 \\ 
\hline

FR12 & Functional & The system must support geographic search for air quality data by country, city, or region. & US15 \\ 
\hline

FR13 & Functional & The system must provide a responsive web interface that works correctly on mobile, tablet, and desktop devices. & US16 \\ 
\hline

FR14 & Functional & \textit{(Optional)} The system may allow users to share links to selected air quality views or reports on social media platforms, using simple shareable URLs. & US17 \\
\hline

% Non-functional requirements (NFR1–NFR10) updated according to instructor feedback.
% See _temp_corrections/NFR_changes_spec.md

NFR1 & Non-Functional & Queries on air quality data (for typical usage scenarios and expected data volumes) must execute in under 2 seconds for at least 95\% of requests, under normal load conditions. & US3, US12 \\
\hline

NFR2 & Non-Functional & The system must support automated periodic ingestion of air quality data without manual intervention, according to a configurable schedule (e.g., every few minutes or hourly), instead of requiring continuous streaming 24/7. & US1, US14 \\ 
\hline

NFR3 & Non-Functional & Data storage must be dimensioned, indexed, and organized to handle the expected volume of readings and users over the project's time horizon, and to allow future partitioning if needed. & US1, US3 \\ 
\hline

NFR4 & Non-Functional & Customized reports with filters on date, location, and indicators must be generated in under 10 seconds for typical workloads. & US5 \\ 
\hline

NFR5 & Non-Functional & The recommendation logic (for simple rule-based recommendations) should be updated at least every 10 minutes, aligned with the update frequency of the external air quality APIs used by the system. & US8, US10 \\ 
\hline

NFR6 & Non-Functional & Air quality data views and visualizations must load in less than 2 seconds for at least 95\% of user requests, under normal conditions. & US12 \\ 
\hline

NFR7 & Non-Functional & The system architecture must provide basic fault tolerance, ensuring that no permanent data loss occurs in case of a single node or instance failure, using regular backups and recovery procedures. Geographic redundancy across multiple regions is explicitly out of scope for the course project. & US14 \\ 
\hline

NFR8 & Non-Functional & The system should be designed so that it can scale horizontally in the future (e.g., by adding read replicas or sharding), but a multi-node deployment is not required for the baseline course implementation (considered a future scalability goal). & US13, US14 \\ 
\hline

NFR9 & Non-Functional & The system's web user interface must function correctly on all major browsers and operating systems commonly used by the target audience, without critical errors. & US16 \\ 
\hline

NFR10 & Non-Functional & Data consistency and user personalization (e.g., preferences and dashboard configuration) must be preserved across all user devices, assuming that users log in with the same account. & US16 \\ 
\hline

\caption{Requirements}
\end{longtable}

\subsubsection*{Non-Functional Requirements Justification}

\textbf{NFR2 (Automated periodic ingestion):} Air quality monitoring requires regular automated data collection since pollution levels fluctuate throughout the day. Periodic ingestion (e.g., every few minutes or hourly) aligns with the update frequency of external APIs while maintaining data freshness for timely detection of hazardous conditions.

\textbf{NFR4 (10 seconds report generation):} Complex reports with multiple filters and large datasets require processing time, but 10 seconds maintains user engagement while allowing for comprehensive data analysis.

\textbf{NFR5 (10 minutes update frequency):} Air quality APIs from external sources typically update their information every 10 minutes to 1 hour, making more frequent updates unnecessary and resource-intensive.

\textbf{NFR6 (Page and visualization load time):} Critical safety information like air quality alerts must be delivered quickly through responsive UI loading to enable timely user decisions about outdoor activities and health precautions.

\textbf{NFR7 (Basic fault tolerance):} While multi-region geographic redundancy is beyond the scope of this course project, basic fault tolerance through regular backups and recovery procedures ensures data integrity and system reliability at a reasonable level for an academic implementation.

\subsubsection*{Future Functional Requirements}

The following functional requirements are considered beyond the baseline scope of this course and may be implemented in future phases:

\begin{itemize}
    \item \textbf{FR11:} In a future phase, the system may provide a basic map-based visualization that highlights areas with better or worse air quality using simple overlays on a web map. (Associated User Stories: US11)
\end{itemize}

\newpage