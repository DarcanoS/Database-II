\section{Requirements}

% Functional requirements (FR1–FR14) updated according to instructor feedback – see Requirements_FR_corrections.csv

\begin{longtable}{|p{2cm}|p{2cm}|p{8cm}|p{3cm}|}
\hline
\textbf{ID} & \textbf{Type} & \textbf{Requirement} & \textbf{Associated User Stories} \\
\hline
\endfirsthead

\hline
\textbf{ID} & \textbf{Type} & \textbf{Requirement} & \textbf{Associated User Stories} \\
\hline
\endhead

FR1 & Functional & The system must periodically collect up-to-date air quality data from multiple external sources (e.g., APIs and monitoring stations) and store it for further processing. & US1, US13, US14 \\ 
\hline

FR2 & Functional & The system must allow users to query and visualize historical air quality data filtered by date range, location, and pollutant type. & US2, US7 \\ 
\hline

FR3 & Functional & The system must display air quality information in a consistent and clear format, independently of the original data source. & US1, US3 \\ 
\hline

FR4 & Functional & The system must display dashboards with key air quality indicators (e.g., AQI, main pollutant, daily trends) based on the latest available data. & US4 \\ 
\hline

FR5 & Functional & The system must generate custom reports with filters for date range, location, and selected indicators, and allow users to download these reports in standard formats. & US5 \\ 
\hline

FR6 & Functional & The system must present graphs showing the evolution of air quality over time, allowing users to interact with basic controls such as selecting date ranges or toggling pollutants. & US6 \\ 
\hline

FR7 & Functional & The system must allow users to export historical air quality data in standard formats such as CSV and JSON, based on the filters applied in the user interface. & US7 \\ 
\hline

FR8 & Functional & The system must provide simple, rule-based recommendations based on the user's location and current air quality conditions (e.g., avoiding outdoor exercise, using a mask). & US8 \\ 
\hline

FR9 & Functional & The system must send air quality alerts when configurable thresholds are exceeded, according to each user's notification preferences. & US9 \\ 
\hline

FR10 & Functional & \textit{(Optional)} The system may suggest appropriate protective measures and product categories (e.g., certified masks, air purifiers) during high pollution periods, as informational guidance only. & US10 \\ 
\hline

FR12 & Functional & The system must support geographic search for air quality data by country, city, or region. & US15 \\ 
\hline

FR13 & Functional & The system must provide a responsive web interface that works correctly on mobile, tablet, and desktop devices. & US16 \\ 
\hline

FR14 & Functional & \textit{(Optional)} The system may allow users to share links to selected air quality views or reports on social media platforms, using simple shareable URLs. & US17 \\ 
\hline

NFR1 & Non-Functional & Queries on large datasets (≥1 million records) must execute in under 2 seconds 95\% of the time. & US3, US12 \\ 
\hline

NFR2 & Non-Functional & The system must support continuous streaming data ingestion 24/7 without manual intervention. & US1, US14 \\ 
\hline

NFR3 & Non-Functional & Data storage must be distributed and optimized for big data processing. & US1, US3 \\ 
\hline

NFR4 & Non-Functional & Customized reports must be generated in under 10 seconds. & US5 \\ 
\hline

NFR5 & Non-Functional & The recommendation engine should be updated every 10 minutes throughout the day with air quality data. & US8, US10 \\ 
\hline

NFR6 & Non-Functional & Air quality data and visualizations must load in less than 2 seconds for 95\% of user requests. & US12 \\ 
\hline

NFR7 & Non-Functional & The system architecture must include fault tolerance and geographic redundancy. & US14 \\ 
\hline

NFR8 & Non-Functional & The system must scale horizontally to support growth in data volume and users. & US13, US14 \\ 
\hline

NFR9 & Non-Functional & The system UI must function correctly on all major browsers and operating systems without errors. & US16 \\ 
\hline

NFR10 & Non-Functional & Data consistency and user personalization must be preserved across all user devices. & US16 \\ 
\hline

\caption{Requirements}
\end{longtable}

\subsubsection*{Non-Functional Requirements Justification}

\textbf{NFR2 (24/7 streaming ingestion):} Air quality monitoring requires continuous data collection since pollution levels fluctuate throughout the day and immediate detection of hazardous conditions is critical for public health.

\textbf{NFR4 (10 seconds report generation):} Complex reports with multiple filters and large datasets require processing time, but 10 seconds maintains user engagement while allowing for comprehensive data analysis.

\textbf{NFR5 (10 minutes update frequency):} Air quality APIs from external sources typically update their information every 10 minutes to 1 hour, making more frequent updates unnecessary and resource-intensive.

\textbf{NFR6 (2 seconds data loading):} Critical safety information like air quality alerts must be delivered quickly to enable timely user decisions about outdoor activities and health precautions.

\subsubsection*{Future Functional Requirements}

The following functional requirements are considered beyond the baseline scope of this course and may be implemented in future phases:

\begin{itemize}
    \item \textbf{FR11:} In a future phase, the system may provide a basic map-based visualization that highlights areas with better or worse air quality using simple overlays on a web map. (Associated User Stories: US11)
\end{itemize}

\newpage