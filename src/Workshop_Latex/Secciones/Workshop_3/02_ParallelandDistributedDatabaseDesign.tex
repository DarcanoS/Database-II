\section{Parallel and Distributed Database Design}

\subsection{Context and Scope of This Section}

This section provides an analytical and forward-looking exploration of how the Air Quality Analysis Platform could evolve into a parallel and distributed database model as its operational scale increases. Currently, the platform relies on a centralized architecture that combines PostgreSQL with TimescaleDB extensions and a complementary NoSQL store. This setup is sufficient for its present data volume, geographic coverage, and analytical needs.

However, as the number of monitored locations grows and long-term retention of real-time geospatial measurements increases, the underlying infrastructure may require structural adjustments. The purpose of this section is therefore not to describe the current implementation, but to outline possible future design paths that ensure long-term scalability, reliability, and performance.

\subsection{Rationale for Future Distributed or Parallel Architectures}

Several projected developments of the platform may demand a transition toward distributed or parallel database mechanisms:

\begin{itemize}[leftmargin=1.5em]
    \item \textbf{Increasing ingestion frequency and scale:} Data is ingested every 10 minutes from multiple external APIs. As more stations and cities are added, the volume of incoming measurements will increase considerably, generating higher write throughput.
    
    \item \textbf{Growth of historical time-series datasets:} Storing several years of air quality and geospatial data results in large, continuously expanding tables, intensifying analytical workloads over time.
    
    \item \textbf{More complex analytical and BI queries:} Public agencies, researchers, and institutions may require long-range trend evaluations, spatial joins, or aggregated models that can benefit from parallel execution.
    
    \item \textbf{Geographic expansion of the system:} Integrating additional regions introduces challenges such as increased latency, larger query spans, and potential regulatory constraints regarding data locality.
    
    \item \textbf{High availability and operational continuity:} Since the platform supports real-time dashboards, notifications, and decision-making applications, uninterrupted service becomes essential, encouraging the adoption of replicated and fault-tolerant environments.
\end{itemize}

In this context, adopting a distributed or parallel database model in the future would support:

\begin{itemize}[leftmargin=1.5em]
    \item \textbf{Horizontal scalability} for both ingestion and analytical processes.
    \item \textbf{Lower latency} through geographically localized data nodes.
    \item \textbf{Separation of workloads} between ingestion, storage, and business intelligence layers.
    \item \textbf{Higher availability} through replication and automated failover mechanisms.
\end{itemize}

\subsection{Potential Distribution and Fragmentation Approaches}

The following strategies describe how the platform’s data could be partitioned or distributed across multiple nodes as the system grows:

\subsubsection{Temporal Fragmentation (Time-Based Partitioning Across Nodes)}

Air quality data is inherently chronological. A natural distribution approach would be to fragment data by time ranges (e.g., monthly or quarterly segments), placing older partitions on separate nodes. This approach supports:

\begin{itemize}[leftmargin=1.5em]
    \item Faster queries over recent data (typically most requested).
    \item Distributed processing for long historical analyses.
    \item Efficient archival and data-retention policies.
\end{itemize}

\subsubsection{Geographical Fragmentation (Region-Based Distribution)}

Since the system manages air quality readings by city and country, data could be distributed by geographic zones. Benefits include:

\begin{itemize}[leftmargin=1.5em]
    \item Lower latency for regional queries.
    \item Improved performance for geospatial filters and location-based analytics.
    \item Compliance with potential future regional data regulations.
    \item Reduced load on the primary cluster during regional peak usage.
\end{itemize}

\subsubsection{Hybrid Fragmentation Model (Time + Geography)}

A combined model could be applied when the density of readings increases significantly. For example:

\begin{itemize}[leftmargin=1.5em]
    \item Partitions by country distributed across regional nodes.
    \item Within each region, temporal partitions optimized for time-series operations.
\end{itemize}

This hybrid model aligns with the platform’s long-term objective of supporting multi-city and multi-country audiences efficiently.

\subsubsection{Replication for High Availability}

To maintain uninterrupted service, read replicas could be introduced for:

\begin{itemize}[leftmargin=1.5em]
    \item Dashboards and public-facing visualizations.
    \item Analytical workloads executed by BI tools or researchers.
\end{itemize}

This ensures that ingestion processes do not compete with visualization or reporting tasks, improving both reliability and performance.

\subsubsection{Parallel Execution for Analytical Queries}

As historical datasets grow, analytic operations—such as aggregations, pollutant correlation analysis, and geospatial pattern detection—could benefit from:

\begin{itemize}[leftmargin=1.5em]
    \item Parallel execution plans within PostgreSQL clusters.
    \item Distributed compute nodes dedicated to research-intensive workloads.
\end{itemize}

These enhancements strengthen the platform’s ability to support institutional and scientific stakeholders.

\subsection{Summary}

Although the platform currently operates under a centralized database architecture, projected growth in data volume, geospatial coverage, and analytical requirements makes it necessary to anticipate more sophisticated storage and processing models. The distribution strategies outlined—temporal, geographic, hybrid, and workload-based—provide feasible migration paths that maintain alignment with the platform’s objectives of scalability, high availability, and low-latency access.
\newpage
