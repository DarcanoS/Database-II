\begin{abstract}
Air pollution remains the second leading cause of premature death worldwide, with disproportionate impacts on Latin American megacities such as Bogotá. We propose a cloud-ready architecture designed to integrate ten-minute data streams from environmental sensors and third-party APIs (AQICN, Google Air Quality, and IQAir) using a Python Ingestor with raw storage in MinIO. A lightweight normalization pipeline transforms heterogeneous payloads before inserting records into a monthly-partitioned PostgreSQL database extended with TimescaleDB hypertables. Query acceleration through concurrently refreshed materialized views supports an API layer (REST + GraphQL) that will serve real-time dashboards and personalized health recommendations. The system targets p95 latencies below 2 seconds under 1000 concurrent users, with an initial deployment covering three years (2022–2024) of Bogotá air quality data.
\end{abstract}

\begin{IEEEkeywords}
Air Quality Index (AQI); TimescaleDB; PostgreSQL partitioning; MinIO; Materialized Views; Personalized Recommendation.
\end{IEEEkeywords}