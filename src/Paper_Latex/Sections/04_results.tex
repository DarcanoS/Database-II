% -------------------------------------------------
\section{Expected Results}\label{sec:results}
% -------------------------------------------------
Table~\ref{tab:targets} presents the target performance and quality metrics aligned with the non-functional requirements (NFR1–NFR8) defined in the project specification.  
These thresholds will be validated during planned load testing with 1000 concurrent users (US13) once the implementation phase is complete.

\begin{table}[tb]
\centering
\caption{Target performance and quality metrics mapped to non-functional requirements.}
\label{tab:targets}
\begin{tabular}{lcc}
\toprule
\textbf{Metric} & \textbf{Target (p95)} & \textbf{Requirement}\\
\midrule
Query latency (≥1M rows)      & $\le2$ s        & NFR1, NFR3\\
Dashboard load time           & $\le2$ s        & NFR6, US12\\
Report generation             & $\le10$ s       & NFR4\\
Recommendation update freq.   & $10$ min        & NFR5\\
Materialized view refresh     & $\le5$ s        & Near-real-time\\
Concurrent user capacity      & $1000$ users    & US13, NFR8\\
System uptime                 & $\ge99.9\%$     & NFR7, US14\\
Peak CPU utilization          & $<70\%$         & Headroom\\
\bottomrule
\end{tabular}
\end{table}

The evaluation methodology will include:
\begin{itemize}
    \item \textbf{Query performance}: Execution time measurements over partitioned tables containing 1+ million air quality records filtered by city, date, and pollutant type (NFR1).
    \item \textbf{Dashboard responsiveness}: End-to-end latency from API request to visualization rendering under normal and peak load conditions (NFR6).
    \item \textbf{Scalability}: Apache JMeter load tests simulating 1000 concurrent users accessing dashboards, generating reports, and triggering alerts simultaneously (US13, NFR8).
    \item \textbf{Data freshness}: Monitoring of ingestion-to-availability lag for the 10-minute update cycle from external APIs (NFR5, US1).
    \item \textbf{Availability}: Uptime tracking and fault tolerance validation through simulated node failures (NFR7, US14).
\end{itemize}

Future iterations of this paper will present measured results including throughput-versus-concurrency curves, ingestion lag distribution over 24-hour periods, and comparison against baseline PostgreSQL performance without TimescaleDB optimizations.
