\chapter{Discussion and Analysis}
\label{ch:evaluation}

This chapter interprets the planned architecture and expected performance metrics presented in Chapter~\ref{ch:results}, discusses their significance relative to the project objectives, and identifies key limitations and areas for future improvement.

%%%%%%%%%%%%%%%%%%%%%%%%%%%%%%%%%%%%%%%%%%%%%%%%%%%%%%%%%%%%%%%%%%%%%%%%%%%%%%%%%%%
\section{Architectural Design Decisions}
\label{sec:architecture_discussion}

The choice of TimescaleDB-augmented PostgreSQL over distributed streaming frameworks (Kafka/Flink/Spark) represents a deliberate trade-off between operational complexity and performance requirements. For the Bogotá single-city deployment with 10-minute ingestion intervals, TimescaleDB's monthly-partitioned hypertables and continuous aggregates provide sufficient query performance without requiring multi-node cluster management, complex stream topology design, or distributed consensus protocols.

\textbf{Strengths of the chosen approach:}
\begin{itemize}
    \item \textbf{SQL familiarity}: PostgreSQL's mature ecosystem and standard SQL interface reduce the learning curve for developers and researchers querying air quality data.
    \item \textbf{Operational simplicity}: Single-node deployment (with optional read replicas) avoids the operational overhead of coordinating distributed brokers, stream processors, and state stores.
    \item \textbf{Time-series optimization}: TimescaleDB's automatic chunk management, native time-based partitioning, and continuous aggregates are purpose-built for time-series workloads and eliminate the need for manual partition maintenance.
    \item \textbf{Cost efficiency}: Leveraging open-source components (PostgreSQL, TimescaleDB, MinIO, Grafana) minimizes licensing costs and enables flexible deployment on cloud or on-premises infrastructure.
\end{itemize}

\textbf{Trade-offs and constraints:}
\begin{itemize}
    \item \textbf{Scaling limits}: The single-node architecture is suitable for Bogotá's 10-minute update cycle but may require re-architecting for sub-minute ingestion or multi-city deployments with thousands of stations.
    \item \textbf{Stream processing}: The current batch-oriented ingestion pipeline lacks the real-time windowing, watermark handling, and exactly-once semantics provided by dedicated stream processors like Flink.
    \item \textbf{Fault tolerance}: While PostgreSQL supports replication and failover, distributed systems like Kafka offer stronger guarantees for message durability and replay in the event of data corruption or node failures.
\end{itemize}

The planned load testing with 1000 concurrent users will validate whether the chosen architecture meets performance targets (NFR1--NFR8). If p95 latency exceeds 2 seconds under production load, mitigation strategies include read replicas for dashboard traffic separation, Redis caching for frequently accessed queries, and incremental adoption of message queues (RabbitMQ or Kafka) for fully asynchronous ingestion pipelines.

%%%%%%%%%%%%%%%%%%%%%%%%%%%%%%%%%%%%%%%%%%%%%%%%%%%%%%%%%%%%%%%%%%%%%%%%%%%%%%%%%%%
\section{Significance of the Findings}
\label{sec:significance}

The air quality monitoring platform addresses a critical public health need in Bogotá, where PM2.5 concentrations frequently exceed WHO guidelines and contribute to respiratory disease burden. By integrating data from multiple authoritative sources (AQICN, Google Air Quality, IQAir) and providing personalized, explainable health recommendations, the system empowers citizens to make informed decisions about outdoor activities, protective equipment, and exposure risk.

\textbf{Key contributions:}
\begin{itemize}
    \item \textbf{Data integration}: Harmonizing heterogeneous API schemas, pollutant units, and AQI scales into a unified relational model enables consistent cross-source queries and reduces citizen confusion from conflicting air quality reports.
    \item \textbf{Scalability validation}: The planned 1000-concurrent-user load test demonstrates the system's capacity to serve a metropolitan population (Bogotá: $\sim$8 million residents) under realistic dashboard access patterns.
    \item \textbf{Explainable recommendations}: Rule-based health advice mapped from EPA AQI bands provides transparency and traceability, avoiding the "black box" problem of machine learning models while remaining interpretable by non-technical users.
    \item \textbf{Audit and replay}: Raw JSON payloads archived in MinIO enable reprocessing with updated normalization rules, bug fixes, or alternative AQI calculation methods without data loss.
\end{itemize}

The 10-minute recommendation update cycle aligns with WHO guidance that short-term exposure reductions can mitigate acute health effects. Future machine learning extensions (ARIMA, LSTM for PM2.5 forecasting) could enable predictive alerts hours before pollution spikes, further improving health outcomes.

%%%%%%%%%%%%%%%%%%%%%%%%%%%%%%%%%%%%%%%%%%%%%%%%%%%%%%%%%%%%%%%%%%%%%%%%%%%%%%%%%%%
\section{Limitations and Challenges}
\label{sec:limitations}

Despite the system's strengths, several limitations affect interpretation and generalization of the results:

\subsection{Data Quality and Sensor Calibration}
External APIs aggregate data from heterogeneous sensor networks, including low-cost sensors that may introduce bias or drift. The platform assumes provider-side calibration and does not implement on-device calibration procedures. Systematic sensor errors could propagate through the normalization pipeline and affect recommendation accuracy.

\textbf{Mitigation strategies:}
\begin{itemize}
    \item Cross-validation: compare readings from multiple co-located sensors and flag outliers.
    \item Reference station anchoring: calibrate low-cost sensors against high-precision government monitoring stations.
    \item Temporal consistency checks: detect and quarantine readings that violate physical plausibility (e.g., negative concentrations, impossible PM2.5 spikes).
\end{itemize}

\subsection{Recommendation Engine Limitations}
The rule-based recommendation engine maps AQI categories to generic health advice templates and does not incorporate individual medical history, pre-existing conditions, or medication interactions. Recommendations should be treated as informational guidance, not clinical directives.

\textbf{Future enhancements:}
\begin{itemize}
    \item Collaboration with public health authorities to validate rule thresholds against local epidemiological data.
    \item Optional user profiles for respiratory conditions (asthma, COPD) to customize sensitivity thresholds.
    \item Integration with electronic health records (subject to privacy regulations) for personalized risk scoring.
\end{itemize}

\subsection{Single-City Deployment Scope}
The current architecture targets Bogotá only and has not been validated for multi-city or international deployments with different AQI standards, regulatory frameworks, or data privacy requirements.

\textbf{Scalability considerations:}
\begin{itemize}
    \item Geographic partitioning by city in addition to temporal partitioning by month.
    \item Regional API endpoints to reduce cross-continent latency for global deployments.
    \item Localization of health recommendations to account for cultural differences in risk perception and communication preferences.
\end{itemize}

\subsection{Performance Assumptions}
The target metrics (NFR1--NFR8) assume a hardware profile of 4 vCPU, 16 GB RAM for the database node. Results may differ on constrained hardware or cloud instances with different I/O characteristics (network-attached storage vs. local NVMe SSDs).

%%%%%%%%%%%%%%%%%%%%%%%%%%%%%%%%%%%%%%%%%%%%%%%%%%%%%%%%%%%%%%%%%%%%%%%%%%%%%%%%%%%
\section{Implications for Practice and Research}
\label{sec:implications}

The platform demonstrates that time-series database optimizations (partitioning, continuous aggregates, BRIN indexes) can deliver near-real-time performance for citizen-facing air quality dashboards without requiring distributed stream processing infrastructure. This finding has practical implications for resource-constrained municipalities and environmental agencies seeking to deploy monitoring systems with limited IT budgets and operational expertise.

From a research perspective, the hybrid architecture (batch ingestion + materialized views) provides a foundation for future studies on:
\begin{itemize}
    \item Query optimization techniques for multi-dimensional time-series data (temporal, spatial, pollutant type).
    \item Trade-offs between view materialization strategies (eager vs. lazy refresh, full vs. incremental updates).
    \item Machine learning integration for forecasting and anomaly detection on pre-aggregated historical data.
\end{itemize}

%%%%%%%%%%%%%%%%%%%%%%%%%%%%%%%%%%%%%%%%%%%%%%%%%%%%%%%%%%%%%%%%%%%%%%%%%%%%%%%%%%%
\section{Summary}
\label{sec:discussion_summary}

This chapter discussed the rationale behind key architectural decisions (TimescaleDB vs. distributed streaming), interpreted the significance of expected performance metrics relative to public health goals, and identified limitations related to data quality, recommendation generalizability, and single-city deployment scope. The planned load testing will validate whether the chosen design meets performance targets and inform future scalability improvements for multi-city deployments and predictive analytics extensions.