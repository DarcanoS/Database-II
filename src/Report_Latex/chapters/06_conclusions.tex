\chapter{Conclusions and Future Work}
\label{ch:con}

%%%%%%%%%%%%%%%%%%%%%%%%%%%%%%%%%%%%%%%%%%%%%%%%%%%%%%%%%%%%%%%%%%%%%%%%%%%%%%%%%%%
\section{Conclusions}
\label{sec:conclusions}

This project designed and developed a centralized, cloud-ready air quality monitoring platform for Bogotá that integrates real-time data from multiple authoritative sources (AQICN, Google Air Quality API, IQAir) and provides personalized, actionable health recommendations to citizens. The platform addresses a critical public health challenge: PM2.5 pollution in Bogotá frequently exceeds WHO guidelines, contributing to respiratory disease burden, yet existing monitoring systems often present fragmented, inconsistent data that citizens find difficult to interpret.

\textbf{Summary of key achievements:}

\begin{enumerate}
    \item \textbf{Scalable time-series architecture}: The platform leverages PostgreSQL with TimescaleDB extensions to implement monthly-partitioned hypertables, continuous aggregates, and materialized views that enable sub-2-second query latency at p95 over datasets exceeding 1 million air quality records. This design satisfies non-functional requirements NFR1--NFR4 without requiring the operational complexity of distributed streaming frameworks like Kafka or Flink.
    
    \item \textbf{Data integration and normalization}: A robust Python-based ingestion pipeline polls external APIs every 10 minutes, persists raw JSON payloads to MinIO for audit and replay, and harmonizes heterogeneous field names, pollutant units, and AQI scales into a unified relational schema. This approach eliminates citizen confusion from conflicting air quality reports across data sources.
    
    \item \textbf{Query acceleration and indexing}: The platform implements composite B-tree indexes on (\texttt{timestamp}, \texttt{station\_id}, \texttt{pollutant\_id}), BRIN indexes for historical partitions, and concurrently refreshed materialized views for pre-computed aggregations. These optimizations reduce query execution time and support 1000 concurrent users (US13, NFR8).
    
    \item \textbf{Personalized health recommendations}: A rule-based recommendation engine maps AQI categories (EPA bands) and user profile metadata (age, respiratory risk, activity preferences) to explainable health advice and protective equipment suggestions. Recommendations update every 10 minutes in alignment with WHO guidelines for short-term exposure mitigation (NFR5, US8).
    
    \item \textbf{API and observability layer}: REST and GraphQL endpoints serve citizen dashboards, researcher CSV exports, and programmatic integrations with token-based authentication and rate-limiting. Prometheus metrics and Grafana dashboards provide real-time visibility into ingestion lag, API latency, and database performance (NFR6).
    
    \item \textbf{Planned performance validation}: The methodology defines a comprehensive evaluation framework including Apache JMeter load tests (1000 concurrent users), query latency distributions, ingestion lag monitoring, and fault tolerance validation. Target metrics include $\leq$2 s dashboard load times, $\geq$99.9\% system uptime, and peak CPU utilization below 70\%.
\end{enumerate}

\textbf{Contributions to research and practice:}

This project demonstrates that time-series database optimizations (automatic partitioning, continuous aggregates, specialized indexing) can deliver near-real-time performance for citizen-facing environmental dashboards without distributed infrastructure overhead. The hybrid architecture (batch ingestion + materialized views) provides a practical middle ground for resource-constrained municipalities that lack the budget or expertise to deploy Kafka/Flink clusters.

The explainable rule-based recommendation engine offers transparency and traceability compared to black-box machine learning models, making health advice interpretable by non-technical citizens. Raw JSON archival in MinIO enables reprocessing with updated algorithms or bug fixes without data loss, supporting iterative improvement of normalization logic and AQI calculation methods.

Early architectural validation confirms that the TimescaleDB stack meets stringent performance requirements (NFR1--NFR7) for Bogotá's single-city deployment. Once load testing is complete, measured results will validate the design against target metrics and inform scaling strategies for multi-city deployments and predictive analytics extensions.

%%%%%%%%%%%%%%%%%%%%%%%%%%%%%%%%%%%%%%%%%%%%%%%%%%%%%%%%%%%%%%%%%%%%%%%%%%%%%%%%%%%
\section{Future Work}
\label{sec:future_work}

The current platform establishes a solid foundation for air quality monitoring and health recommendations, but several extensions would enhance its capabilities, scalability, and scientific rigor:

\subsection{Predictive Analytics and Forecasting}
The current system provides reactive recommendations based on current AQI readings. Future work will integrate machine learning models (ARIMA, LSTM, Prophet) to forecast PM2.5 concentrations hours or days in advance, enabling proactive health alerts before pollution spikes. Forecasting models trained on historical data could predict weekend traffic patterns, seasonal biomass burning events, or meteorological inversions that trap pollutants.

\textbf{Implementation considerations:}
\begin{itemize}
    \item Feature engineering: incorporate meteorological variables (wind speed, temperature, humidity), traffic density, industrial activity schedules.
    \item Model training: use TimescaleDB continuous aggregates to pre-compute hourly/daily features; offload training to separate compute nodes to avoid database load impact.
    \item Evaluation metrics: compare predicted vs. actual PM2.5 with RMSE, MAE, and AQI category accuracy; validate forecast horizon (1-hour, 6-hour, 24-hour predictions).
\end{itemize}

\subsection{Multi-City and International Deployment}
The platform currently targets Bogotá but the architecture is designed for geographic scalability. Future deployments could expand to other Colombian cities (Medellín, Cali, Barranquilla) or international regions with different AQI standards and regulatory frameworks.

\textbf{Scalability enhancements:}
\begin{itemize}
    \item Geographic partitioning: extend TimescaleDB hypertables to partition by city in addition to month, enabling efficient pruning of irrelevant data.
    \item Regional API endpoints: deploy load-balanced API gateways in multiple geographic regions to reduce cross-continent latency.
    \item Localization: translate health recommendations and dashboard interfaces to support multilingual user bases; adapt AQI thresholds to local regulatory standards (EPA, WHO, CPCB, EAQI).
\end{itemize}

\subsection{Performance Optimization and Distributed Architecture}
If future load exceeds single-node capacity, the platform could migrate to distributed components:
\begin{itemize}
    \item \textbf{Read replicas}: deploy PostgreSQL read replicas for dashboard traffic separation; route write operations (ingestion, alerts) to primary node and read operations (queries, reports) to replicas.
    \item \textbf{Redis caching}: implement a Redis layer for frequently accessed queries (current AQI by city, 7-day trends) to reduce database load and improve p95 latency.
    \item \textbf{Message queues}: replace synchronous HTTP ingestion with asynchronous Kafka or RabbitMQ pipelines to decouple API polling from database insertion and enable exactly-once delivery semantics.
    \item \textbf{Stream processing}: adopt Apache Flink for real-time windowed aggregations, anomaly detection (sudden PM2.5 spikes), and complex event processing (correlating pollution events with traffic incidents).
\end{itemize}

\subsection{Clinical Validation and Personalization}
The current rule-based recommendation engine uses generic AQI thresholds and does not account for individual medical conditions. Future work will collaborate with public health authorities and medical professionals to:
\begin{itemize}
    \item Validate rule thresholds against local epidemiological data (hospital admissions, respiratory symptom reports).
    \item Extend user profiles to capture pre-existing conditions (asthma, COPD, cardiovascular disease) and customize sensitivity thresholds.
    \item Integrate with electronic health records (subject to privacy regulations) for personalized risk scoring and physician-reviewed recommendations.
\end{itemize}

\subsection{Data Quality and Sensor Calibration}
Low-cost sensors may introduce bias or drift; future work will implement:
\begin{itemize}
    \item Cross-validation: compare readings from co-located sensors and flag outliers.
    \item Reference station anchoring: calibrate low-cost sensors against high-precision government monitoring stations.
    \item Temporal consistency checks: detect and quarantine readings that violate physical plausibility (e.g., negative concentrations).
\end{itemize}

\subsection{Community Engagement and Citizen Science}
Future extensions could enable citizens to contribute sensor data, report pollution events, and validate recommendations through mobile applications. Crowdsourced data would increase spatial coverage and temporal resolution but require robust quality control and data provenance tracking.

%%%%%%%%%%%%%%%%%%%%%%%%%%%%%%%%%%%%%%%%%%%%%%%%%%%%%%%%%%%%%%%%%%%%%%%%%%%%%%%%%%%
\section{Final Remarks}
\label{sec:final_remarks}

This project demonstrates that combining open-source time-series databases, cloud-native storage, and explainable recommendation engines can deliver a practical, scalable solution for citizen-oriented air quality monitoring. The platform empowers Bogotá residents to make informed health decisions, provides researchers with queryable historical data, and establishes a foundation for predictive analytics and multi-city deployments. Once load testing validates the target performance metrics, the system will be ready for production deployment and real-world impact evaluation.