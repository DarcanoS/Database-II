%Two resources useful for abstract writing.
% Guidance of how to write an abstract/summary provided by Nature: https://cbs.umn.edu/sites/cbs.umn.edu/files/public/downloads/Annotated_Nature_abstract.pdf %https://writingcenter.gmu.edu/guides/writing-an-abstract
\chapter*{\center \Large  Abstract}
%%%%%%%%%%%%%%%%%%%%%%%%%%%%%%%%%%%%%%
% Replace all text with your text
%%%%%%%%%%%%%%%%%%%%%%%%%%%%%%%%%%%

Air pollution remains the second leading cause of premature death worldwide, with disproportionate impacts on Latin American megacities such as Bogotá. This report presents a comprehensive cloud-ready architecture designed to integrate real-time air quality data from multiple sources and provide personalized health recommendations to citizens. The system collects ten-minute data streams from environmental sensors and third-party APIs (AQICN, Google Air Quality, and IQAir) using a Python Ingestor with raw storage in MinIO. A lightweight normalization pipeline transforms heterogeneous payloads before inserting records into a monthly-partitioned PostgreSQL database extended with TimescaleDB hypertables for efficient time-series operations. Query acceleration is achieved through concurrently refreshed materialized views that support an API layer (REST + GraphQL) designed to serve real-time dashboards and personalized health recommendations. The architecture includes a recommendation engine that translates AQI thresholds and user metadata into color-coded health advice and protective product suggestions during high pollution periods. The system targets p95 latencies below 2 seconds under 1000 concurrent users, with an initial deployment covering three years (2022–2024) of Bogotá air quality data. This work addresses the critical gap between raw environmental data availability and actionable citizen-oriented decision support, demonstrating how modern database technologies can be leveraged to improve public health outcomes in urban environments.

%%%%%%%%%%%%%%%%%%%%%%%%%%%%%%%%%%%%%%%%%%%%%%%%%%%%%%%%%%%%%%%%%%%%%%%%%s
~\\[1cm]
\noindent % Provide your key words
\textbf{Keywords:} Air Quality Index (AQI), TimescaleDB, PostgreSQL partitioning, MinIO, Materialized Views, Personalized Recommendation

\vfill
\noindent
\textbf{Report's total word count:} Approximately 10,000-15,000 words (starting from Chapter 1 and finishing at the end of the conclusions chapter, excluding references, appendices, abstract, text in figures, tables, listings, and captions). \newline
\newline
\noindent
\textbf{Source Code Repository:} \url{https://github.com/DarcanoS/Database-II} \newline
\newline
This report was submitted as part of the Databases II course requirements at Francisco José de Caldas District University, Faculty of Engineering, Systems Engineering Program.

