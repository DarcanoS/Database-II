%Two resources useful for abstract writing.
% Guidance of how to write an abstract/summary provided by Nature: https://cbs.umn.edu/sites/cbs.umn.edu/files/public/downloads/Annotated_Nature_abstract.pdf %https://writingcenter.gmu.edu/guides/writing-an-abstract
\chapter*{\center \Large  Abstract}
%%%%%%%%%%%%%%%%%%%%%%%%%%%%%%%%%%%%%%
% Replace all text with your text
%%%%%%%%%%%%%%%%%%%%%%%%%%%%%%%%%%%

Air pollution continues to be a major public-health challenge, particularly in large Latin American cities such as Bogotá. This report describes a practical and reproducible architecture and data design for integrating periodic air-quality data from multiple providers, storing normalized records in a relational database, and delivering clear, rule-based health recommendations to users. The baseline implementation centers on a PostgreSQL-based analytical schema (with temporal partitioning and materialized views where appropriate) combined with a lightweight NoSQL store for user preferences and dashboard configuration. A periodic Python ingestion pipeline normalizes heterogeneous payloads and performs batched inserts aligned with temporal partitions. The API layer follows REST principles and serves dashboards and recommendation endpoints; the recommendation logic is rule-based and maps AQI thresholds and basic user metadata to actionable health guidance. More advanced components—such as dedicated object storage for raw payloads, time-series extensions, and external BI platforms—are discussed as future work. The report focuses on data normalization, indexing strategies, and query optimization decisions that support near real-time dashboards and analytical reporting for urban air-quality monitoring.

%%%%%%%%%%%%%%%%%%%%%%%%%%%%%%%%%%%%%%%%%%%%%%%%%%%%%%%%%%%%%%%%%%%%%%%%%s
~\\[1cm]
\noindent % Provide your key words
	extbf{Keywords:} Air Quality Index (AQI), PostgreSQL partitioning, Materialized Views, Data Normalization, Rule-based Recommendations

\vfill
\noindent
\textbf{Report's total word count:} Approximately 10,000-15,000 words (starting from Chapter 1 and finishing at the end of the conclusions chapter, excluding references, appendices, abstract, text in figures, tables, listings, and captions). \newline
\newline
\noindent
\textbf{Source Code Repository:} \url{https://github.com/DarcanoS/Database-II} \newline
\newline
This report was submitted as part of the Databases II course requirements at Francisco José de Caldas District University, Faculty of Engineering, Systems Engineering Program.

