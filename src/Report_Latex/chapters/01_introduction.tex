\chapter{Introduction}
\label{ch:into}

Air quality monitoring has become a critical concern for public health, particularly in rapidly growing urban centers. This report presents the design and implementation of a practical architecture for near real-time (periodic) air quality monitoring and personalized health recommendations, with a specific focus on Bogotá, Colombia.

%%%%%%%%%%%%%%%%%%%%%%%%%%%%%%%%%%%%%%%%%%%%%%%%%%%%%%%%%%%%%%%%%%%%%%%%%%%%%%%%%%%
\section{Background}
\label{sec:into_back}

Ambient and household air pollution jointly kill an estimated seven to eight million people every year, with 99\% of the world's population breathing air that exceeds WHO guideline values. The 2024 State of Global Air report ranks PM$_{2.5}$ exposure as the second leading risk factor for mortality, ahead of high blood pressure or smoking. In Bogotá, long-term analyses still show spatially heterogeneous PM$_{2.5}$ hot spots, despite a gradual decline from 15.7~$\mu$g/m$^3$ in 2017 to 13.1~$\mu$g/m$^3$ in 2019 and targeted reductions after the city's Air Plan 2030.

While multiple data sources for air quality information exist, including AQICN (minute-level AQI for over 100 countries), Google Air Quality API (500m-resolution indices), and IQAir AirVisual (calibrated sensor networks), these platforms present significant limitations. They provide raw values without personalization, enforce strict quota limits that complicate city-scale analytics, and generally aggregate data hourly with tiered pricing for higher-volume access. Citizens seeking actionable information face the challenge of navigating multiple fragmented sources, understanding technical indicators, and interpreting how air quality conditions specifically affect their health and daily activities.

This project addresses the gap between abundant raw environmental data and the lack of intuitive, timely, and personalized access to air quality information. By integrating heterogeneous data sources into a unified platform with near real-time dashboards, personalized recommendations, and health alerts, the system aims to empower citizens with actionable insights that support daily decisions regarding outdoor activities and health precautions.

%%%%%%%%%%%%%%%%%%%%%%%%%%%%%%%%%%%%%%%%%%%%%%%%%%%%%%%%%%%%%%%%%%%%%%%%%%%%%%%%%%%
\section{Problem statement}
\label{sec:intro_prob_art}

Despite the availability of multiple air quality data sources, citizens in Bogotá face significant challenges in accessing intuitive and rapid information about air pollution levels and their health implications. Current platforms require users to:

\begin{itemize}
    \item Navigate multiple fragmented data sources with inconsistent formats and units
    \item Interpret technical indicators (PM$_{2.5}$, PM$_{10}$, O$_3$, NO$_2$, etc.) without clear health guidance
    \item Manually correlate air quality levels with personal health conditions and activities
    \item Access information through platforms that lack real-time updates or personalization
    \item Understand complex technical documentation without citizen-oriented interfaces
\end{itemize}

This fragmentation and lack of personalization creates a barrier between valuable environmental data and actionable health decisions, particularly affecting vulnerable populations such as children, elderly individuals, and people with respiratory conditions who would benefit most from timely, personalized air quality guidance.

%%%%%%%%%%%%%%%%%%%%%%%%%%%%%%%%%%%%%%%%%%%%%%%%%%%%%%%%%%%%%%%%%%%%%%%%%%%%%%%%%%%
\section{Aims and objectives}
\label{sec:intro_aims_obj}

	extbf{Aim:} The primary aim of this project is to design and implement a centralized, practical air quality monitoring platform that integrates periodic data from multiple sources and provides personalized, actionable health recommendations to citizens in Bogotá.

\textbf{Objectives:} To achieve this aim, the following specific objectives were established:

\begin{enumerate}
    \item Design a scalable database architecture using PostgreSQL (with declarative temporal partitioning and materialized views where appropriate) to efficiently store and query time-series air quality data.
    \item Implement a data ingestion pipeline that collects data from external APIs (AQICN, Google Air Quality, IQAir) at periodic intervals (e.g., 10-minute cycles) and persists normalized records in PostgreSQL; object storage for raw payloads is considered an optional future enhancement.
    \item Develop a normalization process that transforms heterogeneous data formats into a unified schema with monthly partitioning for optimal query performance.
    \item Use materialized views and targeted refresh strategies to accelerate frequent aggregations and support low-latency dashboards as a design goal.
    \item Design and implement a REST API layer to serve dashboards and recommendation endpoints; GraphQL is identified as a possible extension for complex client needs.
    \item Develop a rule-based recommendation engine that maps AQI thresholds and basic user metadata to concise health guidance.
    \item Define and document a performance validation plan with target latency goals and load scenarios; measurements and tuning are part of the evaluation strategy rather than assumed achievements.
    \item Document the system architecture, implementation choices, and lessons learned to guide future scalability and multi-region considerations.
\end{enumerate}



%%%%%%%%%%%%%%%%%%%%%%%%%%%%%%%%%%%%%%%%%%%%%%%%%%%%%%%%%%%%%%%%%%%%%%%%%%%%%%%%%%%
\section{Solution approach}
\label{sec:intro_sol}

The solution adopts a modern, cloud-ready architecture based on PostgreSQL as the core data store, extended with TimescaleDB for efficient time-series operations. The approach follows a systematic pipeline architecture that addresses data ingestion, normalization, storage, and delivery:

\subsection{Data Collection and Ingestion}
\label{sec:intro_ingestion}

A Python-based ingestion service collects air quality data from primary external APIs (for example: AQICN, Google Air Quality, and IQAir) at periodic intervals (a 10-minute cycle is used in the prototype as an illustrative example). Incoming payloads are normalized and inserted into the relational schema in PostgreSQL using batched operations aligned with temporal partitions. Storing raw payloads in an object store is documented as an optional approach for auditability and reprocessing in future iterations, but it is not a required component of the baseline implementation.

\subsection{Database Architecture and Query Optimization}
\label{sec:intro_database}

The normalized data is stored in PostgreSQL using temporal partitioning (monthly partitions are recommended in the prototype) and a set of targeted indexes and materialized views to accelerate common analytical queries. Time-series extensions and advanced compression strategies are discussed as future enhancements; the baseline focuses on declarative partitioning, careful index selection, and precomputed summaries as practical techniques to balance write throughput and read performance for near real-time dashboards.

\subsection{API Layer and User Services}
\label{sec:intro_api}

The baseline API layer follows REST principles and exposes endpoints for dashboard rendering and recommendation retrieval. GraphQL is proposed as an extension for clients that require more flexible querying. The recommendation logic is implemented as rule-based routines that combine AQI-derived thresholds with minimal user metadata (location, declared health conditions) to produce concise health guidance. Advanced features such as product recommendations or navigation guidance are documented as future work and are not part of the baseline.


%%%%%%%%%%%%%%%%%%%%%%%%%%%%%%%%%%%%%%%%%%%%%%%%%%%%%%%%%%%%%%%%%%%%%%%%%%%%%%%%%%%
\section{Summary of contributions and achievements}
\label{sec:intro_sum_results}

This project contributes a practical, reproducible baseline architecture and an accompanying set of engineering decisions for urban air-quality monitoring. The contributions are:

- A documented PostgreSQL-based analytical schema and ingestion pipeline that unifies heterogeneous public APIs into a normalized relational model, including guidance on temporal partitioning, unique constraints and indexes suited for typical analytical queries.
- A rule-based recommendation approach that maps AQI thresholds to concise, actionable health guidance tailored by minimal user metadata.
- A pragmatic performance and validation plan that describes how to measure and tune query latency and ingestion throughput using materialized views, targeted indexing, and batching strategies (measurements are part of the evaluation work).

The prototype is evaluated on a Bogotá dataset spanning recent years (2022–2024) and is presented as a basis for future extensions such as dedicated object storage for raw payloads, time-series extensions, richer BI/visualization stacks, predictive modeling, and multi-region deployments.

%%%%%%%%%%%%%%%%%%%%%%%%%%%%%%%%%%%%%%%%%%%%%%%%%%%%%%%%%%%%%%%%%%%%%%%%%%%%%%%%%%%
\section{Organization of the report}
\label{sec:intro_org}

The rest of this report is organised as follows:

\begin{enumerate}
    \item Chapter~\ref{ch:lit_rev} presents a comprehensive literature review covering air quality monitoring systems, time-series database technologies, and real-time data processing architectures relevant to environmental health informatics;
    \item Chapter~\ref{ch:method} describes the methodology and system design, including the database schema, partitioning strategies, materialized view implementation, API architecture, and recommendation engine logic;
    \item Chapter 4 illustrates the implementation details, including the Python ingestion service, normalization pipeline, database configuration, and API endpoints;
    \item Chapter 5 presents the testing and validation approach, including performance benchmarking, load testing methodology, and compliance verification against functional and non-functional requirements;
    \item Chapter 6 discusses the results, analyzing query performance, system scalability, and effectiveness of the recommendation engine;  and
    \item Finally, Chapter~\ref{ch:con} concludes the report, summarizing key findings, discussing limitations, and outlining future work including predictive modeling capabilities and multi-region deployment strategies.
\end{enumerate}

