\chapter{Introduction}
\label{ch:into}

Air quality monitoring has become a critical concern for public health, particularly in rapidly growing urban centers. This report presents the design and implementation of a comprehensive architecture for real-time air quality monitoring and personalized health recommendations, with a specific focus on Bogotá, Colombia.

%%%%%%%%%%%%%%%%%%%%%%%%%%%%%%%%%%%%%%%%%%%%%%%%%%%%%%%%%%%%%%%%%%%%%%%%%%%%%%%%%%%
\section{Background}
\label{sec:into_back}

Ambient and household air pollution jointly kill an estimated seven to eight million people every year, with 99\% of the world's population breathing air that exceeds WHO guideline values. The 2024 State of Global Air report ranks PM$_{2.5}$ exposure as the second leading risk factor for mortality, ahead of high blood pressure or smoking. In Bogotá, long-term analyses still show spatially heterogeneous PM$_{2.5}$ hot spots, despite a gradual decline from 15.7~$\mu$g/m$^3$ in 2017 to 13.1~$\mu$g/m$^3$ in 2019 and targeted reductions after the city's Air Plan 2030.

While multiple data sources for air quality information exist, including AQICN (minute-level AQI for over 100 countries), Google Air Quality API (500m-resolution indices), and IQAir AirVisual (calibrated sensor networks), these platforms present significant limitations. They provide raw values without personalization, enforce strict quota limits that complicate city-scale analytics, and generally aggregate data hourly with tiered pricing for higher-volume access. Citizens seeking actionable information face the challenge of navigating multiple fragmented sources, understanding technical indicators, and interpreting how air quality conditions specifically affect their health and daily activities.

This project addresses the gap between abundant raw environmental data and the lack of intuitive, rapid, and personalized access to air quality information. By integrating heterogeneous data sources into a unified platform with real-time dashboards, personalized recommendations, and health alerts, the system aims to empower citizens with actionable insights that directly support daily decision-making regarding outdoor activities, health precautions, and protective measures.

%%%%%%%%%%%%%%%%%%%%%%%%%%%%%%%%%%%%%%%%%%%%%%%%%%%%%%%%%%%%%%%%%%%%%%%%%%%%%%%%%%%
\section{Problem statement}
\label{sec:intro_prob_art}

Despite the availability of multiple air quality data sources, citizens in Bogotá face significant challenges in accessing intuitive and rapid information about air pollution levels and their health implications. Current platforms require users to:

\begin{itemize}
    \item Navigate multiple fragmented data sources with inconsistent formats and units
    \item Interpret technical indicators (PM$_{2.5}$, PM$_{10}$, O$_3$, NO$_2$, etc.) without clear health guidance
    \item Manually correlate air quality levels with personal health conditions and activities
    \item Access information through platforms that lack real-time updates or personalization
    \item Understand complex technical documentation without citizen-oriented interfaces
\end{itemize}

This fragmentation and lack of personalization creates a barrier between valuable environmental data and actionable health decisions, particularly affecting vulnerable populations such as children, elderly individuals, and people with respiratory conditions who would benefit most from timely, personalized air quality guidance.

%%%%%%%%%%%%%%%%%%%%%%%%%%%%%%%%%%%%%%%%%%%%%%%%%%%%%%%%%%%%%%%%%%%%%%%%%%%%%%%%%%%
\section{Aims and objectives}
\label{sec:intro_aims_obj}

\textbf{Aim:} The primary aim of this project is to design and implement a centralized, cloud-ready air quality monitoring platform that integrates real-time data from multiple sources and provides personalized, actionable health recommendations to citizens in Bogotá.

\textbf{Objectives:} To achieve this aim, the following specific objectives were established:

\begin{enumerate}
    \item Design a scalable database architecture using PostgreSQL with TimescaleDB extensions to efficiently store and query time-series air quality data from multiple sources.
    \item Implement a data ingestion pipeline that collects data from external APIs (AQICN, Google Air Quality, IQAir) at 10-minute intervals with raw storage in MinIO.
    \item Develop a normalization process that transforms heterogeneous data formats into a unified schema with monthly partitioning for optimal query performance.
    \item Implement materialized views with concurrent refresh capabilities to accelerate queries and support sub-2-second response times under high load.
    \item Design and implement a REST + GraphQL API layer to serve real-time dashboards and support diverse client applications.
    \item Develop a recommendation engine that translates AQI thresholds and user metadata into personalized health advice and protective product suggestions.
    \item Validate the system architecture against performance requirements including p95 latencies below 2 seconds under 1000 concurrent users.
    \item Document the complete system architecture, implementation details, and lessons learned for future scalability and multi-region deployments.
\end{enumerate}



%%%%%%%%%%%%%%%%%%%%%%%%%%%%%%%%%%%%%%%%%%%%%%%%%%%%%%%%%%%%%%%%%%%%%%%%%%%%%%%%%%%
\section{Solution approach}
\label{sec:intro_sol}

The solution adopts a modern, cloud-ready architecture based on PostgreSQL as the core data store, extended with TimescaleDB for efficient time-series operations. The approach follows a systematic pipeline architecture that addresses data ingestion, normalization, storage, and delivery:

\subsection{Data Collection and Ingestion}
\label{sec:intro_ingestion}

A Python-based ingestion service operates on a 10-minute interval cycle, collecting air quality data from three primary external APIs: AQICN, Google Air Quality, and IQAir. Raw payloads are immediately stored in MinIO object storage to preserve original data for audit trails and potential reprocessing. This two-stage approach ensures data durability while allowing the normalization pipeline to operate asynchronously.

\subsection{Database Architecture and Query Optimization}
\label{sec:intro_database}

The normalized data is stored in a PostgreSQL database utilizing monthly partitioning strategies to optimize query performance for time-range filters. TimescaleDB hypertables provide automatic time-series optimizations, including compression and continuous aggregates. To achieve the target sub-2-second query latency, the system employs concurrently refreshed materialized views that pre-compute common aggregations and summary statistics. This multi-layered storage strategy balances write throughput for real-time ingestion with read performance for user queries and dashboard rendering.

\subsection{API Layer and User Services}
\label{sec:intro_api}

A dual-protocol API layer supports both REST endpoints for simple queries and GraphQL for complex, nested data requirements. This flexibility allows mobile applications, web dashboards, and third-party integrations to efficiently access the data they need. The recommendation engine operates as a separate service that consumes air quality data alongside user profiles (location, activity patterns, health conditions) to generate personalized health advice, protective product suggestions, and cleaner-area navigation guidance during high pollution periods.


%%%%%%%%%%%%%%%%%%%%%%%%%%%%%%%%%%%%%%%%%%%%%%%%%%%%%%%%%%%%%%%%%%%%%%%%%%%%%%%%%%%
\section{Summary of contributions and achievements}
\label{sec:intro_sum_results}

This project makes three primary contributions to the field of environmental data systems and public health informatics. First, it demonstrates an end-to-end PostgreSQL-based architecture that successfully unifies three heterogeneous public APIs through a systematic ingestion, normalization, and storage pipeline. The implementation of TimescaleDB hypertables with monthly city-based partitioning provides a practical model for handling large-scale time-series environmental data with efficient query performance.

Second, the lightweight recommendation engine bridges the gap between raw environmental metrics and citizen-actionable guidance. By translating technical AQI thresholds and pollutant concentrations into color-coded health advice, protective product suggestions, and cleaner-area navigation, the system addresses user stories that emphasize personalized health support rather than merely data visualization.

Third, the architecture targets specific performance metrics designed to meet non-functional requirements including sub-2-second query response times at the 95th percentile under 1000 concurrent users. The use of concurrently refreshed materialized views demonstrates how modern database features can be leveraged to achieve real-time responsiveness without sacrificing data accuracy or resorting to complex distributed streaming frameworks.

The initial deployment focuses on three years (2022-2024) of historical Bogotá air quality data, providing a foundation for future expansion to predictive modeling, multi-region deployments, and integration with additional health and environmental data sources.

%%%%%%%%%%%%%%%%%%%%%%%%%%%%%%%%%%%%%%%%%%%%%%%%%%%%%%%%%%%%%%%%%%%%%%%%%%%%%%%%%%%
\section{Organization of the report}
\label{sec:intro_org}

The rest of this report is organised as follows:

\begin{enumerate}
    \item Chapter~\ref{ch:lit_rev} presents a comprehensive literature review covering air quality monitoring systems, time-series database technologies, and real-time data processing architectures relevant to environmental health informatics;
    \item Chapter~\ref{ch:method} describes the methodology and system design, including the database schema, partitioning strategies, materialized view implementation, API architecture, and recommendation engine logic;
    \item Chapter 4 illustrates the implementation details, including the Python ingestion service, normalization pipeline, database configuration, and API endpoints;
    \item Chapter 5 presents the testing and validation approach, including performance benchmarking, load testing methodology, and compliance verification against functional and non-functional requirements;
    \item Chapter 6 discusses the results, analyzing query performance, system scalability, and effectiveness of the recommendation engine;  and
    \item Finally, Chapter~\ref{ch:con} concludes the report, summarizing key findings, discussing limitations, and outlining future work including predictive modeling capabilities and multi-region deployment strategies.
\end{enumerate}

