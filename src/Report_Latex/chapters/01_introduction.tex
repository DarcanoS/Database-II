\chapter{Introduction}
\label{ch:into}

Air quality monitoring is a critical public-health priority in rapidly growing urban areas. This report presents the design and implementation of a consolidated platform for periodic air-quality data integration and citizen-oriented recommendations, with a focus on Bogotá, Colombia.

%%%%%%%%%%%%%%%%%%%%%%%%%%%%%%%%%%%%%%%%%%%%%%%%%%%%%%%%%%%%%%%%%%%%%%%%%%%%%%%%%%%
\section{Background}
\label{sec:into_back}

Ambient and household air pollution jointly cause an estimated seven to eight million deaths annually, with 99\% of the global population breathing air exceeding WHO guideline values. PM\textsubscript{2.5} is the second leading global risk factor for mortality. In Bogotá, long-term observations continue to reveal spatially heterogeneous PM\textsubscript{2.5} hotspots despite recent improvements supported by initiatives such as the Air Plan 2030.

Several public platforms—AQICN, Google Air Quality, and IQAir—provide air-quality data, yet the information is often fragmented, non-personalized, or limited by quotas and aggregation levels. Users frequently lack simple, actionable insights that help them understand pollution trends and their implications for daily activities.

This project aims to bridge the gap between abundant raw data and the need for clear, timely, and citizen-oriented information. By centralizing external data into a unified structure and providing interpretable dashboards and recommendations, the system enhances access to environmental information for citizens, researchers, and public administrators.

%%%%%%%%%%%%%%%%%%%%%%%%%%%%%%%%%%%%%%%%%%%%%%%%%%%%%%%%%%%%%%%%%%%%%%%%%%%%%%%%%%%
\section{Problem statement}
\label{sec:intro_prob_art}

Despite the availability of environmental data sources, citizens in Bogotá still face challenges when trying to understand air pollution levels and their health implications. Current platforms require users to:

\begin{itemize}
\item Navigate fragmented data sources with inconsistent formats and units.
\item Interpret technical pollutant indicators without contextual explanations.
\item Manually relate pollution levels to personal sensitivities or activities.
\item Depend on platforms that do not provide personalized or city-specific insights.
\end{itemize}

As a result, valuable environmental information is not effectively translated into guidance for daily decisions, particularly for sensitive populations such as children, elderly adults, and people with respiratory conditions.

%%%%%%%%%%%%%%%%%%%%%%%%%%%%%%%%%%%%%%%%%%%%%%%%%%%%%%%%%%%%%%%%%%%%%%%%%%%%%%%%%%%
\section{Aims and objectives}
\label{sec:intro_aims_obj}

\textbf{Aim:}
Design and implement a centralized platform that integrates periodically updated air-quality data from multiple sources and provides interpretable, citizen-focused insights for Bogotá.

\textbf{Objectives:}

\begin{enumerate}
\item Design a relational schema in PostgreSQL to store air-quality readings, stations, pollutants, users, alerts, and reports.
\item Implement a lightweight ingestion pipeline that collects external API data at 10-minute intervals using Python.
\item Normalize heterogeneous API responses into a unified structure suitable for analysis and visualization.
\item Use MongoDB to store user preferences, dashboard configurations, and flexible metadata.
\item Develop a simple REST API to expose data for dashboards and citizen-facing services.
\item Implement a rule-based recommendation module that translates AQI levels into health advice.
\item Provide documentation that supports future scalability, including multi-region expansion and integration of additional data sources.
\end{enumerate}

%%%%%%%%%%%%%%%%%%%%%%%%%%%%%%%%%%%%%%%%%%%%%%%%%%%%%%%%%%%%%%%%%%%%%%%%%%%%%%%%%%%
\section{Solution approach}
\label{sec:intro_sol}

The solution follows a modular architecture built on PostgreSQL as the main operational database and MongoDB for flexible user-related information.

\subsection{Data Collection and Ingestion}

A Python service retrieves air-quality observations every 10 minutes from AQICN, Google Air Quality, and IQAir. After basic validation, the data is structured and inserted into PostgreSQL for storage and later analysis.

\subsection{Database Architecture}

The relational schema organizes data into core domains such as stations, pollutants, measurements, users, alerts, and reports. Indexes and well-structured tables ensure efficient filtering by date, station, and pollutant. MongoDB complements the relational store by managing user preferences, dashboard layouts, and other dynamic configurations.

\subsection{API Layer}

A REST API provides access to the integrated dataset. Endpoints support fetching pollution indicators, recommendations, user preferences, and summary statistics. This layer enables web dashboards and other clients to consume the data in a standardized format.

\subsection{Recommendation Logic}

A rule-based module interprets AQI thresholds and pollutant levels to generate clear, actionable recommendations for different user profiles, emphasizing citizen comprehension over technical detail.

%%%%%%%%%%%%%%%%%%%%%%%%%%%%%%%%%%%%%%%%%%%%%%%%%%%%%%%%%%%%%%%%%%%%%%%%%%%%%%%%%%%
\section{Summary of contributions}
\label{sec:intro_sum_results}

This project’s contributions include:

\begin{itemize}
\item A unified PostgreSQL schema that integrates heterogeneous API sources through periodic ingestion and normalization.
\item A dual-storage model (PostgreSQL + MongoDB) separating structured environmental data from flexible user-centric configuration.
\item Citizen-oriented recommendations that translate pollutant levels into actionable guidance.
\item A modular architecture designed for future expansion to predictive modeling, multi-region deployments, and additional environmental indicators.
\end{itemize}

%%%%%%%%%%%%%%%%%%%%%%%%%%%%%%%%%%%%%%%%%%%%%%%%%%%%%%%%%%%%%%%%%%%%%%%%%%%%%%%%%%%
\section{Organization of the report}
\label{sec:intro_org}

The remainder of the report is organized as follows:

\begin{enumerate}
\item Chapter~\ref{ch:lit_rev}: Literature review on air quality monitoring and environmental data systems.
\item Chapter~\ref{ch:method}: Methodology and system design, including schema decisions, APIs, and recommendation logic.
\item Chapter 4: Implementation details for ingestion, normalization, database structure, and REST endpoints.
\item Chapter 5: Testing and validation of ingestion, data integrity, and dashboard functionality.
\item Chapter~\ref{ch:con}: Conclusions, limitations, and future work directions including multi-region expansion and integration of new data sources.
\end{enumerate}
